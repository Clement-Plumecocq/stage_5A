\chapter{Éléments théoriques sur la méthode de champ de phase} \label{chap:2}
Ce deuxième chapitre résume l'ensemble des notions théoriques liées à la méthode champ de phase. Le couplage avec l'hydrodynamique est présenté puis une description analytique de l'énergie ainsi que les principales hypothèses du modèle sont introduites. 
\section{Présentation des principales méthodes de suivi et de capture d'interface}

Le traitement numérique efficace des interfaces représente un enjeu majeur de la simulation numérique tant les applications faisant intervenir des interfaces sont importantes. On différencie les méthodes de suivi d'interface des méthodes de capture d'interface. Les méthodes de suivi d'interface suivent des marqueurs placés sur l'interface au cours du temps, la position de l'interface est alors explicite. Les méthodes de capture d'interface quant à elle suivent implicitement l'interface au travers de l'évolution d'une fonction couleur. De nombreuses méthodes existent, on présente ici les principales :
\begin{itemize}
	\item[$\bullet$] \textit{\textbf{Volume of fluid (VOF) : }} Cette méthode utilise un maillage fixe découpé en cellule représentant des volumes. On associe alors à chacune de ces cellules une fraction volumique de fluide, cette proportion est alors résolue au cours du temps et la position de l'interface peut être reconstruite. Cette reconstruction a pour désavantage de ne fournir que peu d'informations viables sur l'interface. Cette méthode reste donc peu précise et est également difficile à mettre en \oe uvre en trois dimensions.
	\item[$\bullet$] \textit{\textbf{Méthode Level-Set (LS) : }}	Cette méthode repose sur la résolution implicite de l'interface au travers de la résolution d'une fonction auxiliaire dite fonction ligne de niveau, généralement la distance signée à l'interface. Cette fonction se doit d'admettre une valeur nulle à l'interface, ainsi au travers de la résolution d'une équation d'advection sur cette fonction ligne de niveau, l'interface est résolue. Cette méthode convient pour les problèmes à fort changement topologique mais présente le désavantage d'être non-conservative.
	\item[$\bullet$]\textit{\textbf{Arbitrary Lagrangian-Eulerian (ALE) : }} La méthode repose sur une double description lagrangienne (maillage mobile) et eulérienne (maillage fixe), à chaque itération temporelle, le maillage autour de l'interface est reconstruit pour s'adapter à la forme de l'interface, ainsi chaque maille contient uniquement un fluide. L'ensemble de ces propriétés rend la méthode très précise mais difficile à mettre en \oe uvre en trois dimensions.
	\item[$\bullet$]\textit{\textbf{Front-Tracking (FT) : }} La méthode utilise des marqueurs sans masse positionnés sur l'interface transportée suivant une description lagrangienne sur un maillage eulérien fixe. Ainsi les équations de Navier-Stokes sont résolues sur un maillage fixe tandis que l'équation régissant la position de l'interface est résolue sur un maillage mobile. La principale difficulté réside dans le choix des opérateur de communication entre les deux maillages. Cette méthode nécessite l'implémentation d'algorithme pour les cas de coalescence et rupture d'interface et possède comme désavantage de ne pas être conservative.
\end{itemize}








\section{Méthode champ de phase conservative}
\subsection{Présentation générale}
%Les méthodes de suivi ou de capture d'interface présentées précédemment décrivent toute l'interface comme une discontinuité, cependant il existe un second paradigme traitant l'interface comme une zone de transition continue, on parle alors d'interface diffuse. \\
Le traitement numérique de l'interface présente deux paradigmes : le premier traite l'interface comme une discontinuité. Le second modélise l'interface comme une zone de transition continue, on parle alors d'interface diffuse.
Dans ce second cas l'interface correspond donc à une zone d'épaisseur connue et maîtrisée où cohabitent les deux phases, les gradients à l'interface étant finis le traitement numérique est alors facilité. Ce concept d'interface diffuse date du XIX$^{\text{ème}}$siècle et est introduit par Van Der Walls \cite{rowlinson_translation_1979}.
\begin{figure}[H]
	\centering
	\includegraphics[width=0.5\linewidth]{figure/diffuse_interface}
	\caption{Comparaison entre une interface diffuse (à gauche) et une interface raide (à droite)}
	\label{fig:diffuseinterface}
\end{figure} 
 Le concept d'interface diffuse va gagner l'intérêt de la communauté scientifique avec le développement de la méthode champ de phase. En 1950, Ginzburg et Landau propose une description de l'énergie libre d'un système tenant compte des inhomogénéités spatiales (interfaces) en fonction d'un paramètre d'ordre \cite{landau_physique_1995}. En 1958, Cahn et Hilliard \cite{cahn_free_1958}  introduisent l'utilisation de cette description pour décrire l'évolution de la composition. Ces études sont la base de la méthode champ de phase. En effet la méthode champ de phase est basée sur la représentation d'un système au travers d'une fonctionnelle d'énergie libre, la description diffuse de l'interface et le suivi de paramètres d'ordre $\phi$. La méthode champ de phase est aujourd'hui utilisée dans de nombreux domaines, certains sont présentés en figure \ref{fig:champphase}.
\begin{figure}[H]
	\centering
	\includegraphics[width=0.9\linewidth]{figure/champ_phase}
	\caption[Domaine d'application de la méthode champ de phase]{Domaine d'application de la méthode champ de phase, tirée de \cite{introini_suivi_nodate}}
	\label{fig:champphase}
\end{figure} 
\subsection{Équation de Cahn-Hilliard généralisée}
Dans le cas n-composants on note $\{\phi_i\}_{i=1,..n}$ \footnote{Par soucis de simplification on pourra également utiliser la notation $\{\phi\}$} l'ensemble des paramètres d'ordre du système. Ces paramètres peuvent représenter la concentration, la fraction massique ou d'autres grandeurs.
Comme expliqué précédemment la méthode de champ de phase repose sur le suivi de ces variables. Ces paramètres d'ordre peut être conservés ou non, dans notre étude les paramètres d'ordres sont conservés. De plus on considère un système fermé, ainsi on obtient la contrainte :
\begin{equation}
\sum_{i=1}^n \phi_i =1 \Rightarrow \phi_n =1 - \sum_{i=1}^{n-1} \phi_i
\end{equation} 
Ainsi pour un mélange à $n$ composants, cette loi de fermeture permet de décrire l'ensemble du système en suivant l'évolution d'uniquement $n-1$ composants. Dans certains cas le système peut également être décrit avec des variables non conservées telles que des indicatrices de phases ou des grandeurs liées à des réactions chimiques. Le comportement de ces variables est alors régis par des équations de réaction-diffusion dites d'Allen-Cahn, non résolue dans cette étude. Dans le cadre de variables conservées les équations de Cahn-Hillard pour $n$ composants, avec $i\in \{1,..,n-1 \}$ s'écrivent sous la forme :
\begin{equation}
	\cfrac{\partial \phi_i}{\partial t} + \left(\mathbf{u} \cdot \nabla\right) \phi_i =  \nabla \cdot \left(\sum_{j=1}^{n-1}{\mathcal{M}_{ij}} \nabla\left( \frac{\delta \mathbb{F}}{\delta \phi_j}\right) \right) 
\end{equation}
avec : $\mathcal{M}_{ik}$ les coefficient de la mobilité (paramètre cinétique),  $\phi_i$ le paramètre d'ordre, $\mathbf{u}$ la vitesse et $\mathbb{F}$ une fonctionnelle de Ginzburg-Landeau généralisée \cite{cardon_modelisation_2016} définit tel que : 
 \begin{equation}
\mathbb{F}[\{\phi\}] = \int_{\mathcal{V}}\lambda\tilde{g}^{}(\{\phi\},\mathbf{x},t)+ \sum_{i=1}^{n-1}\sum_{j=1}^{n-1}\cfrac{1}{2}\kappa_{ij}\nabla \phi_i \cdot \nabla \phi_j dV
\end{equation}
Avec $\mathbf{x}$ le vecteur position et $V$ le volume.
Le premier terme représente la densité d'énergie liée aux valeurs locales de composition, traduisant l'équilibre des phases ainsi que leurs existences ou coexistence. Pour deux phases $\alpha$ et $\beta$, on rappelle les conditions données par l'équilibre thermodynamique \cite{kim_phase-field_1999} :
\begin{subequations}
	\label{eq:all}
	\begin{empheq}[left={\empheqlbrace\,}]{align}
		&\lambda\left.\frac{\partial \tilde{g}^{}}{\partial \phi_i}\right|_{\phi_i^{\alpha,eq}} = \lambda\left.\frac{\partial \tilde{g}^{}}{\partial \phi_i}\right|_{\phi_i^{\beta,eq}} = \tilde{\mu}_i^{eq} \\
& 		\lambda\tilde{g}^{\alpha,eq} - \sum_{i=1}^{n-1}\tilde{\mu}_i^{eq}\phi_i^{\alpha,eq} = 	\lambda\tilde{g}^{\beta,eq} - \sum_{i=1}^{n-1}\tilde{\mu}_i^{eq}\phi_i^{\beta,eq}
	\end{empheq}
\end{subequations}
$\tilde{\mu}_i^{eq}$ représente un potentiel de diffusion de l'élément $i$ à l'équilibre.\\
Le second terme représente la contribution des interfaces, le coefficient $\kappa_{ij}$, dit coefficient de gradient, tient compte du coût énergétique engendré par l'interface, par la suite ce paramètre pourra être relié à la tension de surface. \\
%Dans le cas binaire il est possible de décrire géométriquement l'ensemble des variables : 
%\begin{figure}[H]
%	\centering
%	\includegraphics[width=0.45\linewidth]{figure/fig_NRJ}
%	\caption{Description de l'énergie libre pour un système binaire}
%	\label{fig:fignrj}
%\end{figure}
Finalement la dérivée variationnelle de cette fonctionnelle d'énergie libre peut être définie comme un potentiel de diffusion $\tilde{\mu}$: 
\begin{equation}\label{eq_potentiel}
	\frac{\delta \mathbb{F}}{\delta \phi_j} =\lambda \frac{\partial \tilde{g}}{\partial \phi_j} -\sum_{k=1}^{n-1} \kappa_{jk} \Delta \phi_k = \tilde{\mu}_j
\end{equation}
avec $\lambda$ un paramètre d'\textit{upscaling} numérique ajouté pour augmenter l'épaisseur de l'interface. Cet élargissement de l'interface permet d'utiliser des maillages réalistes compte tenu des capacités de calcul actuelles. \\
Le potentiel de diffusion peut être relié au potentiel chimique classique tel que :
%\begin{equation}
%	\tilde{\mu}_i = \frac{\lambda}{V_m}\left(\mu_i - \mu_n\right)
%\end{equation}
\begin{subequations}
	\begin{empheq}[left={\empheqlbrace\,}]{align}
	&\tilde{\mu}_i = \frac{\lambda}{V_m} \hat{\mu}_i \\
		& \hat{\mu}_i = \mu_i - \mu_n
	\end{empheq}
\end{subequations}
Avec $\tilde{\mu}_i$ (en J.m$^{-3}$) représente le potentiel de diffusion volumique de l'élément $i$, $\hat{\mu}_i$ (en J.mol$^{-1}$) le potentiel de diffusion molaire et $V_m$ le volume molaire supposé constant dans tout le système. \footnote{Dans le cadre de ce rapport on note $\tilde{.}$ les grandeurs volumiques}\\
%Le potentiel chimique étant classiquement définit tel que :
%\begin{equation}
%	\mu_i = \left.\frac{\partial G}{\partial n_i}\right|_{P,T,n_{j\neq i }} = \left.\frac{\partial F}{\partial n_i}\right|_{V,T,n_{j\neq i }}
%	  \textrm{                        où           } F = V_m f_0
%\end{equation}
%Avec $F$ (resp. $G$) l'énergie libre d'Helmotz (resp. Gibbs) (en J) et $n_i$ la quantité de matière de l'élément $i$ (en mol). Dans notre cas on se place dans une transformation isobare et isotherme, on privilégiera donc l'énergie de libre de Gibbs.
Dans le cas où $\doubleoverline{\kappa}$ = $\doubleoverline{0}$ on retrouve une équation d'advection-diffusion classique, dans le cas contraire on obtient une équation d'ordre 4. Pour un système binaire le gradient d'energie et le paramètre d'\textit{upscaling} peuvent être déterminés analytiquement. Dans le cas d'une modélisation à n-composants cette approche analytique ne fonctionne plus. Ainsi une des principales difficultés de la mise en place de la méthode champ de phase dans les cas n-aire est l'obtention de ces paramètres.
Dans la suite de ce travail nous utiliserons une proposition de paramétrage introduite par \cite{rasolofomanana_numerical_nodate} et présentée en annexe \ref{ann:parametrage}. Cette paramétrisation permet de déterminer les paramètres d'\textit{upscaling} $\lambda$ et le gradient d'énergie $\bm{\bar{\bar{\kappa}}}$ en fonction des paramètres physiques du système, la tension de surface $\sigma$ et l'épaisseur d'interface $\epsilon$. Ce paramétrage est construit de façon à être consistant vis-a-vis d'un système binaire.
\begin{subequations}
	\label{eq:all}
	\begin{empheq}[left={\empheqlbrace\,}]{align}
	&\bm{\bar{\bar{\kappa}}} = \frac{\sigma \epsilon}{\xi_1 \xi_2}\delta_{ij}
	\label{eq:1} \\
	&\lambda=\frac{\xi_2 \sigma}{2\epsilon\xi_1}
	\label{eq:2}
	\end{empheq}
\end{subequations}
avec $\xi_1 ,\xi_2$ deux constantes dépendantes de la description thermodynamique adoptée, $\delta_{ij}$ le symbole de Kronecker.
\subsection{Couplage avec les équations de Navier-Stokes incompressible}
Dans le cadre de cette étude, les équations de Cahn-Hilliard sont couplées aux équations de conservation de masse et de quantité de mouvement incompressible. Kim J. \cite{kim_phase-field_2012} présente un modèle \textit{one fluid} avec densité variable dans le terme de flottabilité. Les équations de Navier-Stokes s'écrivent sous la forme :
\begin{subequations}
	\label{eq:all}
	\begin{empheq}[left={\empheqlbrace\,}]{align}
	&\nabla \cdot \mathbf{u} = 0\\
	&\rho^* \left (\frac{\partial \mathbf{u}}{\partial t} + (\mathbf{u} \cdot {\nabla})\mathbf{u}\right) = -{\nabla} P +\eta \Delta \mathbf{u}+\sum_{i=1}^{n-1} \tilde{\mu}_i{\nabla} \phi_i + \rho\left(\{\phi_i\}\right) \mathbf{g}
	\end{empheq}
\end{subequations}
avec $\mathbf{u}$ la vitesse, $P$ la pression, $\mathbf{g} = \{ 0,0,-g\}^T $, $\eta$ la viscosité dynamique supposée constante, $\rho^*$ la masse volumique du solvant, $\rho\left(\{\phi_i\}\right)$ une loi de densité fonction du paramètre d'ordre. \\
L'hypothèse du volume molaire constant nous impose que la loi de densité soit une combinaison linéaire des masses volumiques des corps purs que l'on écrit sous la forme : 
\begin{equation}
	\rho\left(\{\phi_i\}\right) = \rho(\phi_n)\left(1+\sum_{i=1}^{n-1}\beta_i \phi_i\right)
\end{equation}
Les paramètres $\beta_i$ sont à déterminer en fonction du système étudié, $\rho(\phi_n)$ correspond à une masse volumique de référence, généralement celle du solvant.
\section{Paysage thermodynamique analytique}
\subsection{Introduction d'un pseudo-grand potentiel}
L'objectif présenté dans \cite{rasolofomanana_numerical_nodate} est d'obtenir une formulation analytique du terme homogène de la fonctionnelle de Ginzburg-Landau. Dans le cas binaire, cette contribution est de la forme d'un double puits, généralement un polynôme de degré 4.
L'objectif est de généraliser ce double puits pour un système n-aire, ainsi on introduit un pseudo grand potentiel correspondant à la hauteur énergétique nécessaire pour changer de minimum d'énergie \cite{cardon_modelisation_2016}.
\begin{equation}
\Omega^{\star} =\Omega - \Omega^{eq} =  {g} - \sum_i \hat{\mu}_i^{eq}\phi_i - \left(  {g}^{eq} -  \sum_i \hat{\mu}_i^{eq}\phi_i^{eq} \right) 
\end{equation}
avec ${g}^{}$ l'énergie libre de Gibbs (J.mol$^{-1}$)\footnote{En utilisant l'hypothèse du volume molaire constant il est possible d'écrire $\tilde{g} = {g}/{V_m}$}, utilisé comme grandeur d'intérêt ici puisque le système est supposé isotherme et isobare.
	\subsection{Application pour un cas diphasique ternaire}
Comme présenté au chapitre \ref{chap:1} le cas d'intérêt de l'étude est le corium. Ce système comprend deux phases à l'équilibre et donc deux points d'équilibre distinct.  Ainsi \cite{rasolofomanana_numerical_nodate} introduit une formulation sous la forme d'un double puits de la forme :
\begin{equation}\label{double_puit}
	\Omega^{\star}  = P^{dis} \times P^{cont}
\end{equation}
Où $P^{dis}, P^{cont}$ représentent deux paraboloïdes correspondant aux deux phases en présence à l'équilibre notées dispersée (ou drop) et continue. Dans le cas ternaire, en considérant les éléments d'intérêt notés miscible ($misc$)  et immiscible ($immi$)  : 
\begin{multline}
P^{k}=\left(\frac{\co{\theta^{k}}(\phi_{misc}-\phi_{misc}^{eq,k}) + \sinus{\theta^{k}}(\phi_{immi}-\phi_{immi}^{eq,k})}{a_{misc}^{k}}\right)^{2}+\\ \left(\frac{-\sinus{\theta^{k}}(\phi_{misc}-\phi_{misc}^{eq,k}) + \co{\theta^{k}}(\phi_{immi}-\phi_{immi}^{eq,k})}{a_{immi}^{k}}\right)^{2}
\label{eq:paraboloid_general_}
\end{multline}
Avec $k = \{disp,cont\}$ la phase (dispersée ou continue), $a_{misc}^k$ (resp. $a_{immi}^k$) le demi-petit (resp. demi-grand) puits, $\theta^k$ l'angle de rotation associé au puits de la phase $k$.\\
On trace alors en exemple le cas le plus simple de paysage :
\begin{figure}[H]
	\centering
	\includegraphics[width=0.6\linewidth]{figure/landscape}
	\caption{Exemple de paysage thermodynamique, la zone bleu représente la zone instable}
	\label{fig:landscape}
\end{figure}
%Les éléments de calculs pour la détermination de la zone instable sont présentés en annexe \ref{ann:stabphase}, cette zone correspond à des compositions pour lesquels le système subit une séparation de phase, ainsi il est primordiale que cette zone n'englobe pas les conditions initiales. Une représentation dans le plan est possible pour le cas ternaire, pour les cas comprenant un nombre plus important de paramètres d'ordre la visualisation semble plus complexe. Ainsi cette formulation possède l'avantage d'éviter un couplage entre le code CFD et un solveur d'équilibre thermodynamique (Open-Calphad par exemple) réduisant significativement le temps de développement. En effet pour les cas ou le paysage est connu il est alors possible de calibrer les paramètres des paraboloïdes pour obtenir une forme analytique proche du paysage réel. Dans le cas d'un paysage inconnu, par manque d'information thermodynamique, ce paysage permet d'avoir une description cohérente pour des simulations qualitatives.

La zone instable correspond à la zone où la séparation de phase a lieu, la détermination de cette zone est obtenue d'après \cite{aursand_spinodal_2017}. Une formulation analytique de cette forme permet d'éviter un couplage entre solveur d’équilibre thermodynamique (Open-Calphad par exemple) et le code CFD. Le temps de développement est alors fortement réduit. De plus, dans le cas d'un système sans données thermodynamiques, la formulation permet d'obtenir des simulations qualitatives. Dans le cas d'un système complètement décrit thermodynamiquement il suffit de trouver la formulation analytique qui convient (via un \textit{fit} par exemple). 


%On peut dès lors calculer le potentiel de diffusion homogène grâce à la formulation analytique (\ref{double_puit}) :
%\begin{align}
%	\tilde{\mu}_i & \nonumber= \frac{\partial}{\partial \phi_i}\left\lbrace 
%	\Omega^{\star} + \sum_j \tilde{\mu}_j^{eq}\phi_j + \left( {g}^{liq,eq} -  \sum_j \tilde{\mu}_j^{eq}\phi_j^{eq} \right)\right\rbrace \\
%	&\nonumber = \frac{\partial \Omega^{\star}}{\partial \phi_i} + \frac{\partial g^{liq,eq}}{\partial \phi_i} + \sum_j \frac{\partial \tilde{\mu}_j^{eq}\left(\phi_j - \phi_j^{eq}\right)}{\partial  \phi_i}\\
%	\tilde{\mu}_i &=	P^{dis}\frac{\partial P^{cont}}{\partial \phi_i} + P^{cont}\frac{\partial P^{dis}}{\partial \phi_i} + \tilde{\mu}_i^{eq}
%\end{align} 
%%\begin{align*}
%	%& \frac{\partial g^{liq,eq}}{\partial \phi_i} = 0 \\
%		%& \sum_j \frac{\partial \tilde{\mu}_j^{eq}\left(\phi_j - %\phi_j^{eq}\right)}{\partial  \phi_i} = \tilde{\mu}_i^{eq} %+\tilde{\mu}_{j\neq i}^{eq} \frac{\partial \phi_{j\neq i}}{\partial %\phi_i} = \tilde{\mu}_i^{eq}
%%\end{align*}
%L'objectif est alors de déterminer les paramètres des paraboloïdes pour obtenir des résultats consistants thermodynamiquement.


