\begin{appendix}
\chapter{Solution analytique pour une interface plane}
La solution analytique stationnaire permet de donner des conditions initiales cohérentes, ainsi on s'intéresse au cas binaire "classique" (avec un seul paramètre d'ordre noté $\phi$), La densité d'énergie est choisit de la forme analytique d'ordre 4 en double puits :
\begin{equation}
	f(\phi) = \phi^2 (1-\phi)^2
\end{equation}
La fonctionnelle de Ginzburg-Landau est alors de la forme :
\begin{equation}
	\mathbb{F} =\int_V \lambda f(\phi) + \frac{1}{2}\kappa ||\nabla \phi||^2 dV
\end{equation}
Avec $\lambda$ un paramètre d'upscalling présenté précédemment, $\kappa$ un coefficient de gradient.
Dans l'état stationnaire, la condition d'équilibre nous donne :
\begin{equation}
	\kappa \frac{d^2\phi}{dz^2} = \lambda \frac{d f(\phi)}{d\phi}
\end{equation}
L'astuce est alors de multiplier l'équation par $\displaystyle \frac{d\phi}{dz}$ puis d'intégrer entre 0 et $L$, soit :
\begin{align}
	& \kappa \frac{d^2\phi}{dz^2}\frac{d\phi}{dz} = \lambda \frac{d f(\phi)}{d\phi}\frac{d\phi}{dz} \\
	\Rightarrow & \kappa \int_0^z \frac{d^2\phi}{dz^2}\frac{d\phi}{dz} dz= \int_0^z \frac{d f(\phi)}{dz} dz
\end{align}
Loin de l'interface, en $z=0$ on considère le système à l'équilibre soit $\frac{d\phi}{dz} = 0$ et on fixe une condition au limite de type Dirichlet homogène :
\begin{equation}
	\phi (z= 0) = 0 \Rightarrow f(0) = 0
\end{equation}
Finalement le résultat de l'intégration précédente nous donne :
\begin{equation}
		\frac{1}{2}\kappa \left(\frac{d\phi}{dz}\right) = \lambda f(\phi)
\end{equation}
En remplaçant $f(\phi)$ par sa formulation analytique : 
\begin{equation}
	\frac{d\phi}{\phi(1-\phi)} = \sqrt{\frac{2\lambda}{\kappa}} dz
	\label{solution_statio_plane}
\end{equation}
En posant $u = 2\phi - 1$ soit $du = 2d\phi$
\begin{align*}
	 \cfrac{d\phi}{\phi(1-\phi)} &= \cfrac{\cfrac{1}{2}du}{\left(\cfrac{u+1}{2}\right)\left( 1 -\cfrac{u+1}{2}\right)} \\
	 & = \frac{2du}{(1+u)(1-u)} \\
	 & = 2\frac{du}{1-u^2}
\end{align*}
Finalement l'équation \ref{solution_statio_plane} devient :
\begin{equation}
	\frac{du}{1-u^2} = \cfrac{1}{2}\sqrt{\frac{2\lambda}{\kappa}}dz
\end{equation}
On remarque que le termes de gauche correspond à la dérivée de la fonction réciproque de la tangente hyperbolique, soit :
\begin{equation}
	\text{arctanh}(u) = \cfrac{1}{2}\sqrt{\frac{2\lambda}{\kappa}} z + C
\end{equation}
Avec $C$ une constante d'intégration.
En réutilisant le changement de variable il est immédiat que :
 \begin{equation}
 \phi(z) = \frac{1}{2}\left(\tanh \left(\cfrac{1}{2}\sqrt{\frac{2\lambda}{\kappa}}z +C\right)+1\right)
 \end{equation}
La constante $C$ peut être déterminé à partir d'une valeur moyenne du paramètre d'ordre, cette résolution ne sera pas explicité ici.

\chapter{De la tension de surface au coefficients de gradient}
\section{Méthode générale}
Cette deuxième annexe présente la méthode de calcul des coefficients de gradient à partir de la tension de surface qui est une donnée physique du système. Physiquement la tension interfaciale $\sigma$ représente l'excès d'énergie libre par unité de surface associé à la présence d'une interface entre deux phases distinctes, on peut ainsi la écrire :
\begin{equation}
\sigma = \cfrac{\mathbb{F}-\mathbb{F}^{hom}}{S}
\end{equation}
Où $S$ représente la surface de l'interface, $\mathbb{F}$ l'énergie libre du système et $\mathbb{F}$ l'énergie libre homogène du système, c'est-à-dire l'énergie libre du système dénuée d'interface. 
Dans un premier temps on rappel la forme des fonctionnelles de Ginzburg-Landau associées à ses deux systèmes :
\begin{align*}
&\mathbb{F} = \int_{V}\sum_{i=1}^{n-1}\sum_{j=1}^{n-1}\frac{\kappa_{ij}}{2}\nabla \phi_i \cdot \nabla \phi_j + f(\phi_1,..,\phi_{n-1}) - \sum_{1}^{n-1}\tilde{\mu}_i^{eq}\phi_i dV \\
&\mathbb{F}^{hom} = \int_V f^{\alpha,eq}  - \sum_{1}^{n-1}\tilde{\mu}_i^{eq}\phi_i^{\alpha,eq} dV 
\end{align*}
Ainsi en combinant les deux équations précédentes : 
\begin{equation}
\sigma = \int_{V}\sum_{i=1}^{n-1}\sum_{j=1}^{n-1}\frac{\kappa_{ij}}{2}\nabla \phi_i \cdot \nabla \phi_j + f(\phi_1,..,\phi_{n-1}) - f^{\alpha,eq} - \sum_{1}^{n-1}\tilde{\mu}_i^{eq}(\phi_i-\phi_i^{\alpha,eq}) dV
\end{equation}
Dans le cas 1D suivant $z$ l'équation précédente devient : 
\begin{equation}
\sigma = \int_{0}^L\sum_{i=1}^{n-1}\sum_{j=1}^{n-1}\frac{\kappa_{ij}}{2}\frac{ d\phi_i}{dz} \frac{ d\phi_j}{dz} + f(\phi_1,..,\phi_{n-1}) - f^{\alpha,eq} - \sum_{1}^{n-1}\tilde{\mu}_i^{eq}(\phi_i-\phi_i^{\alpha,eq}) dz
\end{equation}
La condition d'équilibre s'écrit :
\begin{equation}
\sum_{j=1}^{n-1} \kappa_{i,j} \frac{d^2\phi_j}{dz^2} = \left.\frac{\partial f}{\partial \phi_i}\right|_{\phi_{j\neq i}} - \tilde{\mu}_i^{eq}
\end{equation}
Après multiplication par $\displaystyle \frac{d\phi_i}{dz}$ et intégration on trouve :
\begin{equation}
\sigma = \int_0^L \sum_{i=1}^{n-1}\sum_{j=1}^{n-1} \kappa_{i,j}\frac{d\phi_i}{dz}\frac{d\phi_j}{dz}dz
\end{equation}
On retrouve alors la relation permettant la détermination de ce coefficient dans le cas binaire en posant $n=2$
\begin{equation}
\sigma = \int_0^L\kappa^{bin}\left(\frac{d\phi}{dz}\right)^2dz
\end{equation}
Sauf mention contraire, dans le cadre de l'étude on considère :
\begin{equation}
\bm{\bar{\bar{\kappa}}} =    \begin{pmatrix} 
\kappa^{bin}& 0 \\ 
0				& \kappa^{bin} 
\end{pmatrix} 
\end{equation}
\section{Méthode de calcul}



\end{appendix}
