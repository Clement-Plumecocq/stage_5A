\chapter{Séquence de dégradation du c\oe ur} \label{annexe:seq_deg}
D'après \cite{kolev_multiphase_2015} voici le scénario de fonte du c\oe ur :
\begin{itemize}
	\item[$\bullet$] \underline{Entre 800 et 900 \textdegree C :} L'augmentation de la pression à l'intérieur de la gaine en zirconium provoque un gonflement puis une rupture de cette dernière.
	\item[$\bullet$] \underline{Entre 900 et 1300 \textdegree C :} Début de la réaction fortement exothermique d'oxydation de la gaine. A cet instant une forte proportion de la puissance thermique dégagée provient de cette réaction. La molécule d'eau est dissociée, l'oxygène est absorbé par la surface métallique et l'hydrogène est libéré. L'absorption de l'hydrogène dans les fissures du métal le fragilise davantage et accélère le processus de défaillance de la gaine. De plus l'apparition de fissure augmente la surface de réaction provoquant une accélération de la réaction.
	\item[$\bullet$] \underline{Entre 1300 et 1400 \textdegree C :} Apparition d'alliages constitués des matériaux composant la gaine (principalement Zr) et de l'acier présent dans la cuve.
	\item[$\bullet$] \underline{Entre 1400 et 1500 \textdegree C :} Fusion et rupture des structures métallique du c\oe ur. Libération des composant en phase gazeuse du combustible.
	\item[$\bullet$] \underline{Entre 1500 et 1850 \textdegree C :} Point de fusion du zirconium, dissolution du dioxyde d'uranium UO$_2$ par le zirconium en fusion, apparition de l'alliage (U,O,Zr)
	\item[$\bullet$] \underline{Entre 2000 et 2650 \textdegree C :} Fusion du ZrO$_2$, dissolution de l'UO$_2$ dans le ZrO$_2$ fondu et formation de la solution liquide UO$_2$-ZrO$_2$.
\end{itemize}
